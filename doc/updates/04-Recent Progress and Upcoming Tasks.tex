\documentclass[aspectratio=169]{beamer}
	\usepackage[utf8]{inputenc}		% Required for umlauts
	\usepackage[english]{babel}		% Language
	%\usepackage[sfdefault]{roboto}	% Enable sans serif font roboto
	%\usepackage{libertine}			% Enable this on Windows to allow for microtype
	\usepackage[T1]{fontenc}		% Required for output of umlauts in PDF

	\usepackage{mathtools}		% Required for formulas

	\usepackage{caption}		% Customize caption aesthetics
	\usepackage{tcolorbox}		% Fancy colored boxes
	\usepackage{xcolor}			% Highlighting
	\usepackage{soul}

	\usepackage{listings}		% Insert programming code
	\usepackage{graphicx}		% Required to insert images
	\usepackage{subcaption}		% Enable sub-figure
	\usepackage[space]{grffile} % Insert images baring a filename which contains spaces
	\usepackage{float}			% Allow to forcefully set the location of an object

	\usepackage[tracking=true]{microtype} % Required to change character spacing

	\usepackage[style=alphabetic,backend=biber,sorting=none,giveninits=true,isbn=false,url=false]{biblatex}
	\usepackage{csquotes}		% Ensure proper quotation of texts with babel and polyglossia with biblatex
	\usepackage{hyperref}		% Insert clickable references

	\usepackage{datetime}		% Flexible date specification
	\newcommand{\leadingzero}[1]{\ifnum#1<10 0\the#1\else\the#1\fi}
	\newcommand{\todayddmmyyyy}{\leadingzero{\day}.\leadingzero{\month}.\the\year}
	\newcommand{\mathcolorbox}[2]{\colorbox{#1}{$\displaystyle #2$}}

	\usepackage{geometry}
	\usepackage{scrextend}		% Allow arbitrary indentation

	\usepackage{color}

	\usepackage{appendixnumberbeamer}	% Fancy page numbering excluding the appendix

	% Compile notes into a separate file readable by pdfpc using a custom package which overwrite the `note` macro
	\usepackage{../pdfpcnotes}

	\makeatletter
	% Fix subfig in beamer style presentation
	\let\@@magyar@captionfix\relax

	% Insert [short title] for \section in ToC
	\patchcmd{\beamer@section}{{#2}{\the\c@page}}{{#1}{\the\c@page}}{}{}
	% Insert [short title] for \section in Navigation
	\patchcmd{\beamer@section}{{\the\c@section}{\secname}}{{\the\c@section}{#1}}{}{}
	% Insert [short title] for \subsection in ToC
	\patchcmd{\beamer@subsection}{{#2}{\the\c@page}}{{#1}{\the\c@page}}{}{}
	% Insert [short title] for \subsection in Navigation
	\patchcmd{\beamer@subsection}{{\the\c@subsection}{#2}}{{\the\c@subsection}{#1}}{}{}
	\makeatother

	\addbibresource{../literature.bib}

	\setbeamercolor{title}{fg=orange}
	\setbeamertemplate{title}{
		\color{orange}
		\textbf{\inserttitle}
	}
	\setbeamercolor{tableofcontents}{fg=orange}
	\setbeamercolor{section in toc}{fg=black}
	\setbeamercolor{subsection in toc}{fg=black}
	\setbeamertemplate{frametitle}{
		%\vspace{0.5em}
		\color{orange}
		\begin{center}
			\textbf{\insertframetitle} \\
			{\small \insertframesubtitle}
		\end{center}
	}
	\setbeamertemplate{footline}[text line]{
		\parbox{\linewidth}{
			\color{gray}
			\vspace*{-1em}
			NII 2018
			\hfill
			Gordian (\href{mailto:gordian.edenhofer@gmail.com}{gordian.edenhofer@gmail.com})
			\hfill
			\insertframenumber/\inserttotalframenumber%
		}
	}
	\setbeamertemplate{navigation symbols}{}
	\setbeamertemplate{itemize item}{\color{black}$\bullet$}
	\setbeamertemplate{itemize subitem}{\color{black}$\circ$}
	\setbeamercolor{block title}{fg=black}
	\captionsetup{font=scriptsize,labelfont={bf,scriptsize}}

	\lstset{basicstyle=\ttfamily,breaklines=true,showstringspaces=false,commentstyle=\color{red},keywordstyle=\color{blue}}

	\title{Meta-Learning~for~Recommender~Systems}
	\subtitle{``learning to learn''}
	\author[Edenhofer]{\href{mailto:gordian.edenhofer@gmail.com}{Gordian Edenhofer}}
	\institute[NII]{
		Working Group of Prof.~Dr.~Beel, Trinity College Dublin \\
		Department of Prof.~Dr.~Akiko~Aizawa, Nationa Institute of Informatics
	}
	\date[Research Internship 2018]{National Institute of Informatics, \formatdate{06}{09}{2018}}
	\subject{Natural Language Processing and Machine Translation}


\begin{document}

\pagenumbering{arabic}

\begin{frame}[plain,noframenumbering]
	\titlepage%
\end{frame}

\section{Current Progress}
\frame{\vfill\centering\tableofcontents[sectionstyle=show/shaded,subsectionstyle=show/hide]\vfill}

\subsection{Git Log}
\begin{frame}
	\frametitle{\insertsection}
	\framesubtitle{\insertsubsection}

	\begin{itemize}
		\item Implement content-based neural network approach employing fastText
		\item Add further meta-variables
		\begin{itemize}
			\item \#Donations per user
			\item User location information
			\item \ldots
		\end{itemize}
		\item Deploy meta-learners
		\begin{itemize}
			\item Compare a selection of decision tree and ensemble approaches
			\item Implement shallow neural network
			\item Compare classification models to error predicting approaches
		\end{itemize}
	\end{itemize}
\end{frame}

\subsection{Visualization}
\begin{frame}
	\frametitle{\insertsection}
	\framesubtitle{\insertsubsection}

	\begin{figure}
		\centering
		\includegraphics[width=\textwidth,height=0.6\textheight,keepaspectratio]{{{../res/Learning subsystem - Distribution of position in Top-N test set for various algorithms}}}
		\caption{Distribution of the positon of the project to which the user donated to in a set of $100$ projects the user has not donated to. Results are shown for a subset of the learning subsystem.}
	\end{figure}
\end{frame}

\begin{frame}
	\frametitle{\insertsection}
	\framesubtitle{\insertsubsection}

	\begin{figure}
		\centering
		\includegraphics[width=\textwidth,height=0.6\textheight,keepaspectratio]{{{../res/Learning subsystem - Position in Top-N test set for various algorithms}}}
		\caption{Statistical features of the distribution of the donated item position in the Recall@N test set. Hereby, the orange line is indicating the median, the green line the mean.}
	\end{figure}
\end{frame}

\begin{frame}
	\frametitle{\insertsection}
	\framesubtitle{\insertsubsection}

	\begin{figure}
		\centering
		\includegraphics[width=\textwidth,height=0.6\textheight,keepaspectratio]{{{../res/Meta-learner as Classifier and Error Predictor - Average position in Top-N test set for various meta-learner algorithms}}}
		\caption{Performance of various meta-learners in comparison to the overall best algorithm. Meta-learners are used as classifiers (CL) as well as for error predictions (EP).}
	\end{figure}
\end{frame}

\section{Next Steps}
\frame{\vfill\centering\tableofcontents[sectionstyle=show/shaded,subsectionstyle=show/hide]\vfill}

\subsection{TO-DO}
\begin{frame}
	\frametitle{\insertsection}
	\framesubtitle{\insertsubsection}

	\begin{itemize}
		\item Trial-and-error
		\begin{itemize}
			\item Hyperparameter tuning
			\item Exploring the usefulness of sample weights, class weights, \ldots
		\end{itemize}
		\item Add further meta-variables
		\begin{itemize}
			\item Statistical features of the dataset
			\item Pre-calculated values of combinations of meta-features
		\end{itemize}
		\item Explore further collaborative filtering techniques
		\begin{itemize}
			\item Filtering based on common user features, e.g.~location
			\item Possibly implement novel and previously unexplored algorithms (?)
		\end{itemize}
	\end{itemize}
\end{frame}

\end{document}
